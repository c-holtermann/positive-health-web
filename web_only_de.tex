% Spiderweb Diagram
%
% Author: Dominik Renzel
% Author: Christoph Holtermann
% Date; 2009-11-11
% Date; 2022-08-22 modified for positive health
\documentclass[varwidth=80cm]{standalone}
\usepackage{tikz, calc}
\usetikzlibrary{shapes}

\begin{document}

\newcommand{\D}{6} % number of dimensions (config option)
\newcommand{\U}{10} % number of scale units (config option)
\newcommand{\G}{2} % only show every nth web row
\pgfmathtruncatemacro{\UG}{\U/\G} % number of web rows

\newdimen\R % maximal diagram radius (config option)
\R=3.5cm 
\newdimen\L % radius to put dimension labels (config option)
\L=4.4cm

\newcommand{\A}{360/\D} % calculated angle between dimension axes
\newcommand{\anglebase}{\A/2}
\newcommand{\AX}{\X*\A+\anglebase} % calculate angles so that one spike shows down
\newcommand{\labelangle}{\YG*\R/\U}

\begin{figure}[htbp]
 \centering

\begin{tikzpicture}[scale=1]
  \tikzstyle{labelText} = [align=center, font=\small]
  \tikzstyle{labelNumber} = [align=right, text width=1em, font=\small]
  \path (0:0cm) coordinate (O); % define coordinate for origin

  % draw the spiderweb
  \foreach \X in {1,...,\D}{
    \draw (\AX:0) -- (\AX:\R);
  }

  \foreach \Y in {0,...,\UG}{
    \foreach \X in {1,...,\D}{
      \path (\AX:\Y*\G*\R/\U) coordinate (D\X-\Y*\G);
      \fill (D\X-\Y*\G) circle (1pt);
    }

    \pgfmathtruncatemacro{\YC}{\Y*\G}
    \pgfmathsetmacro{\YG}{\Y*\G+1}

    \path (1*\A+\anglebase:\labelangle) node[labelNumber, align=right, xshift=0.05cm, yshift=-0.05cm] {\YC};
    \draw [opacity=0.3] (\anglebase:\Y*\G*\R/\U) \foreach \X in {1,...,\D}{
        -- (\AX:\Y*\G*\R/\U)
    } -- cycle;
  }

  % define labels for each dimension axis (names config option)
  \path (1*\A+\anglebase:\L) node[labelText] (L1) {Körperliche\\Funktionsfähigkeit};
  \path (2*\A+\anglebase:\L) node[labelText] (L2) {Mentale Funktionsfähigkeit\\und Wahrnehmung};
  \path (3*\A+\anglebase:\L) node[labelText] (L3) {Spirituelle/existenzielle\\Dimension};
  \path (4*\A+\anglebase:\L) node[labelText] (L4) {Lebensqualität};
  \path (5*\A+\anglebase:\L) node[labelText] (L5) {Soziale \& gesellschaftliche\\Teilhabe};
  \path (6*\A+\anglebase:\L) node[labelText] (L6) {Alltägliche\\Funktionsfähigkeit};

\end{tikzpicture}
\end{figure}

\end{document}
